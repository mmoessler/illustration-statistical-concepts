\documentclass{article}

% Language setting
% Replace `english' with e.g. `spanish' to change the document language
\usepackage[english]{babel}

% Set page size and margins
% Replace `letterpaper' with `a4paper' for UK/EU standard size
\usepackage[letterpaper,top=2cm,bottom=2cm,left=3cm,right=3cm,marginparwidth=1.75cm]{geometry}

% Useful packages
\usepackage{amsmath}
\usepackage{graphicx}
\usepackage[colorlinks=true, allcolors=blue]{hyperref}

\title{Interactive Illustration of \\ the Sampling Properties of \\ Estimators}
\author{Markus M\"o\ss ler}

\begin{document}
	
\maketitle
	
\begin{abstract}
This is the documentation of the implementation of a learning module for an interactive illustration of the sampling properties of estimators
\end{abstract}

\section{Introduction}

This repository contains the implementation of a learning module with interactive illustrations of fundamental statistical concepts and properties.

\section{Goal of the learning module (Why?)}

Understanding the concept and properties of sampling distributions of estimators is one of the most important concepts of statistical inference, i.e, of learning from data about a (population) model of interest, and thus, a fundamental part of empirical studies in social science in general and econometrics in particular. 
%
Practice, however, has shown that students often have difficulty understanding the concept of sampling distributions and their properties. 
%
One potential reason for this is that the sampling properties of estimators are often formulated and analyzed in an abstract way only. 
%
The goal of this learning module is to give a more intuitive understanding of the sampling properties of estimators using interactive illustrations of simulation results. 

An example is to understand the effect of the sample size $N$ on the sampling properties of the sample average $\overline{X}$ as an estimator for the mean $\mu$ of a random variable of interest as stated in the law of large numbers (LLN) and the central limit theorem (CLT). 
%
Using interactive illustrations of simulation results the students can increase the sample size and observe how the sample average gets closer to the mean (LLN) and how the sampling distribution of the standardized statistic of the sample average gets closer to the standard normal distribution (CLT).

\section{Subject of the learning module (What?)}

\subsection{General}

Subject of this learning module is the sampling properties of different estimators for different specifications of the data generating process (DGP). 
%
The DGP is based on a statistical model with a particular parameter of interest, e.g., the mean of a Bernoulli random variable. 
%
Based on the DGP the parameter of interest is estimated using a particular estimator, e.g., the sample average. 
%
This learning module shows the effect of changes in the DGP on the sampling distribution of an estimator.  

Note, the interactive illustration of the sampling properties of the sample average for increasing the sample size $N$, i.e., the large sample properties of the sample mean, is only one subject of interest. 
%
Other subjects are to understand the effect of omitted variable bias (OVB) and heteroskedasticity on the sampling distribution of the OLS estimator in a simple linear regression model.

\subsection{Bernoulli Distribution and Sample Average}

% ber_dis_sam_ave

\subsection{Continuous Uniform Distribution and Sample Average}

% con_und_dis_sam_ave

\subsection{Cross Section Linear Regression Model and Ordinary Least Squares}

\subsubsection{Parameterization and the Effect of Sample Size}

% cs_lin_reg_ols_01

\subsubsection{Effect of Heteroskedasticity}

% cs_lin_reg_ols_02

\subsubsection{Effect of Omitted Variable Bias}

% cs_lin_reg_ols_03





\section{Method/implementation of the learning module (How?)}





\bibliographystyle{alpha}
\bibliography{bibliography}
	
\end{document}